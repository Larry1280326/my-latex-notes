\documentclass[11pt]{article}
\usepackage[margin=0.8in]{geometry}
\usepackage[]{amsfonts, amssymb, amsmath, float, hyperref,fancyheadings, graphicx, derivative}
\pagestyle{fancy}
\lhead{Honors Design and Analysis of Algorithm}
\chead{Larry128}
\rhead{COMP3711H}
\renewcommand{\footrulewidth}{0.4pt}
\newcommand{\indep}{\perp \!\!\! \perp}
\usepackage{listings}
\usepackage{xcolor}
\definecolor{codegreen}{rgb}{0,0.6,0}
\definecolor{codegray}{rgb}{0.5,0.5,0.5}
\definecolor{codepurple}{rgb}{0.58,0,0.82}
\definecolor{backcolour}{rgb}{0.95,0.95,0.92}
\lstdefinestyle{mystyle}{
    backgroundcolor=\color{backcolour},   
    commentstyle=\color{codegreen},
    keywordstyle=\color{magenta},
    numberstyle=\tiny\color{codegray},
    stringstyle=\color{codepurple},
    basicstyle=\ttfamily\footnotesize,
    breakatwhitespace=false,         
    breaklines=true,                 
    captionpos=b,                    
    keepspaces=true,                 
    numbers=left,                    
    numbersep=5pt,                  
    showspaces=false,                
    showstringspaces=false,
    showtabs=false,                  
    tabsize=3
}
\lstset{style=mystyle}

\begin{document}
\tableofcontents
\section{Runtime analysis of algorithm}
\textbf{Definition of Runtime}\\
We use atomic operations instead of real running time of the algorithm since different devices have different real running time but must have the same atomic operations.\\
Given a set $A$ of atomic operations and an algorithm $A$,
\begin{align*}
T(A, I)&= \text{runtime of A on input I}\\
&= \text{\# atomic operations}
\end{align*}
Assumptions:
\begin{enumerate}
\item $T(A, I)$ only depends on the input $I$, \\\textit{counter-example: coin-flipping doesn't fulfill this assumption.}
\item all the atomic operations are equal, however, some of them may take more time in the real world.
\item $T(A, I)$ only depends on the size of input $I$, denoted as $|I|$, but not a particular input with that size.
\end{enumerate}
\textbf{Worse Case Execution Time (WCET)}
$$T(A, n) = \max_{|I|=n} T(A, I)$$
\textbf{Asymptotic Analysis}
\begin{enumerate}
\item[•] Two algorithm with runtimes $f(n)$, and $g(n)$\\
$5n + 5 = O(n)$, $100n = O(n)$, they are asymptotic equivalent.
\item[•] Formal definition: \\We say $f \in O(g)$ if $$\exists c>0, \forall n \geq n_{0}, f(n) \leq   c g(n)$$
\item[•] Definition 2:\\ we say $f \in \Omega (g)$ if $$\exists n_{0}, \exists c>0, \forall n > n_{0}, c f(n) \geq g(n)$$
\item[•] Definition 3: $$f \in \Theta(g) \iff f\in O(g) \wedge f \in g \in O(f)$$
\item[•] Definition 4: $$f \in o(g) \iff f\in O(g) \wedge f \in g \notin O(f)$$ $$f \in \omega(g) \iff f\in \Omega(g) \wedge f \in g \notin \Omega(f)$$
\end{enumerate}
\textbf{Fibonacci Numbers}\\
$$f_{0} =1, f_{1} = 1, f_{i} = f_{i-1} + f_{i-2}$$
\begin{lstlisting}[language=C++]
int fibo(const int& n){
	if (n==0 || n==1) return 1;
	return fibo(n-1) + fibo(n-2);
}
\end{lstlisting}
$$T(A, n) = \begin{cases} \mbox{$O(1), $}& \mbox{if } n = 0, \text{or } n =1 \\ \mbox{$T(A, n-1) + T(n-2) + O(1)$} & \mbox{otherwise} \end{cases}$$
$$\implies T(n) > f_{n}$$
That is, since $f_{n}$ increase exponentially, the runtime of the algorithm is even worse than the exponential runtime, \textit{which is not desirable}.
Instead of using recursion, we can simply use array
\begin{lstlisting}[language=C++]
int fibo(const int& n){
    int* A = new int[n+1];
    A[0] = A[1] =1;
    for (int i=2; i<n+1; i++){
        A[i] = A[i-1] + A[i-2];
    }
    int result = A[n];
    delete [] A;
    return result;
}
\end{lstlisting}
$$T(A, n) \in O(n)$$
We can see that the latter the algorithm is better since it's faster!\\\\
\textbf{Maximum Contiguous Subsequence}\\
Input: an array $A$ of integers, e.g., $\{10, 5, -2, -3, 10, 12, 0\}$\\
Output: two indices $i$, $j$, such that $\sum_{k=i}^{j}A[k]$ is maximized.\\
There are many algorithms to solve this problem:
\begin{enumerate}
\item By brute force, trying out all the possibilities
\begin{lstlisting}[language = C++]
const int MAXN = 1e6;
int A[MAXN];

int main(){
	int n;
	cin>> n;
	for (int i=0; i<=n; i++){
		cin>> A[i];	
	}
	int ans = 0;
	for (int i=0; i<=n; i++){
		int sum = 0;
		for (int j=i; j<=n; i++){
			sum += A[j];
			if (sum > ans){
				ans = sum;			
			}		
		}
	}
	cout<< ans;
	return 0;
}
\end{lstlisting}
$$T(A, n) \in O(n^{2})$$
\item Divide and conquer
\end{enumerate}
\end{document}